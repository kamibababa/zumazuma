\documentclass[letterpaper]{article}
\begin{document}
\section{Project Proposal}
For the term project, I propose to make a 2-player version of Zuma, where the two players cooperate to eliminate all the balls before they collide into each other. Single-player mode can also be chosen at the beginning of the game.

I propose to use Pygame for the main game structure, and use socket to implement the multiplayer mode, which would also include the \_thread and the queue modules.

One problem that I can imagine now is how to implement the curvy path the string of balls should follow. I see several ways to solve this problem:
\begin{enumerate}
\item\label{method1}I can make the balls behave individually, where they ``jump'' according to a list of coordinates that represent the path. I can write a helper program to acquire the list of coordinates efficiently, and hard-code it into the path implementation.
\item Still making the balls behave independently, I can try to express the paths with trigonometric functions, and calculate the next positions for each of them, provided their current position.
\item\label{method3}I can implement a physical engine, where the balls are bound within walls that I can hard-code into the maps with the method mentioned in \ref{method1}. Meanwhile, ``Gravity'' acts on all the balls and pull them along the path with an invisible and incline, collisions among balls themselves prevent them from reordering in the narrow path.
\item\label{method4}Similar to \ref{method3}, I can implement an attractive force between successive balls, to emulate the backtracking effect when large chunks of balls are eliminated in the original version of Zuma.
\end{enumerate}

Methods \ref{method1}, \ref{method3}, and \ref{method4} all involve hard-coding, but due to the nature of puzzle games, I believe hard-coding maps is most of the times inevitable, and thus acceptable.
\section{Competitive Analysis}
There is already the original version of Zuma, as well as two other versions of Zuma's Revenge! and Zuma Blitz. I think those all have great features, such as sound effects, collision and explosion animations, etc., which I definitely want to incorporate into my project.

I have played another game, BombSquad, and I liked its menu designs, which I will also try to include in my implementation of the game.
\section{Technology Demonstration}
I believe main.py and server.py files serve pretty good as demonstrations of my capability with the module that I wish to use.
\end{document}